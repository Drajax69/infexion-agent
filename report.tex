% Options for packages loaded elsewhere
\PassOptionsToPackage{unicode}{hyperref}
\PassOptionsToPackage{hyphens}{url}
%
\documentclass[
]{article}
\usepackage{amsmath,amssymb}
\usepackage{lmodern}
\usepackage{iftex}
\ifPDFTeX
  \usepackage[T1]{fontenc}
  \usepackage[utf8]{inputenc}
  \usepackage{textcomp} % provide euro and other symbols
\else % if luatex or xetex
  \usepackage{unicode-math}
  \defaultfontfeatures{Scale=MatchLowercase}
  \defaultfontfeatures[\rmfamily]{Ligatures=TeX,Scale=1}
\fi
% Use upquote if available, for straight quotes in verbatim environments
\IfFileExists{upquote.sty}{\usepackage{upquote}}{}
\IfFileExists{microtype.sty}{% use microtype if available
  \usepackage[]{microtype}
  \UseMicrotypeSet[protrusion]{basicmath} % disable protrusion for tt fonts
}{}
\makeatletter
\@ifundefined{KOMAClassName}{% if non-KOMA class
  \IfFileExists{parskip.sty}{%
    \usepackage{parskip}
  }{% else
    \setlength{\parindent}{0pt}
    \setlength{\parskip}{6pt plus 2pt minus 1pt}}
}{% if KOMA class
  \KOMAoptions{parskip=half}}
\makeatother
\usepackage{xcolor}
\IfFileExists{xurl.sty}{\usepackage{xurl}}{} % add URL line breaks if available
\IfFileExists{bookmark.sty}{\usepackage{bookmark}}{\usepackage{hyperref}}
\hypersetup{
  hidelinks,
  pdfcreator={LaTeX via pandoc}}
\urlstyle{same} % disable monospaced font for URLs
\usepackage{longtable,booktabs,array}
\usepackage{calc} % for calculating minipage widths
% Correct order of tables after \paragraph or \subparagraph
\usepackage{etoolbox}
\makeatletter
\patchcmd\longtable{\par}{\if@noskipsec\mbox{}\fi\par}{}{}
\makeatother
% Allow footnotes in longtable head/foot
\IfFileExists{footnotehyper.sty}{\usepackage{footnotehyper}}{\usepackage{footnote}}
\makesavenoteenv{longtable}
\setlength{\emergencystretch}{3em} % prevent overfull lines
\providecommand{\tightlist}{%
  \setlength{\itemsep}{0pt}\setlength{\parskip}{0pt}}
\setcounter{secnumdepth}{-\maxdimen} % remove section numbering
\ifLuaTeX
  \usepackage{selnolig}  % disable illegal ligatures
\fi

\author{}
\date{}

\begin{document}

\hypertarget{introduction}{%
\section{INTRODUCTION}\label{introduction}}

\begin{quote}
Project Part B of COMP30024 was to implement a game playing agent for
the game Infexion. While this was fairly similar to Part A of the
project, the difference primarily is that opponents can make moves and
SPAWN actions are enabled. The two main algorithms considered with Monte
Carlo Tree Search and Minimax.

The algorithm chosen by my team was Minimax, as we felt it had a more
guaranteed success rate compared to Monte Carlo. Due to the large
branching factor and low time and space availability, Monte Carlo may
not have been able to perform enough simulations to get a viable playing
agent. Due to the lack of confidence in the algorithm in this scenario,
Minimax was chosen.
\end{quote}

\hypertarget{agent-algorithm}{%
\section{AGENT ALGORITHM}\label{agent-algorithm}}

Minimax algorithm looks a certain level of optimal moves ahead to choose
the best move based on what the opponents may pick. Despite the large
branching factor, the algorithm was optimised to be able to make moves
in under a second (See Optimisations for more information).

\hypertarget{evaluationutility-function}{%
\subsection{EVALUATION/UTILITY
FUNCTION}\label{evaluationutility-function}}

\begin{quote}
For my algorithm, I have chosen the maximising player to be the colour
RED. The utility function is just a measure of the power levels RED and
BLUE have, described as follows:
\end{quote}

\[U\left( \text{state} \right) = \ power\left( \text{red} \right) - \ power(blue)\]

\hypertarget{dynamic-depth}{%
\subsection{\texorpdfstring{DYNAMIC DEPTH
}{DYNAMIC DEPTH }}\label{dynamic-depth}}

\begin{quote}
My agent can look up to 3 moves ahead really fast, but a depth of 4 can
take a large amount of time if the number of actions (branching factor)
is large. After thorough testing, it was found that \emph{large} is when
the number of actions is greater than 70, where it could take around 20
seconds to make a move. It also depends on the number of moves the
opponent has but abstracting the complexity made it a simpler problem to
deal with.

A few things to consider included the turn count, which essentially
suggested whether we were in early-game (\textless20 moves), mid-game or
late-game. Early-game is less important than mid-game, as most pieces do
not do much capturing early on and the distribution is just based on
spawning. Therefore, the depth searched was also dependent on the turn,
because more the game progresses, more important the moves get.

Another factor that impacts depth search is the amount of time left,
which I broke into a \emph{Panic mode}, \emph{Calm mode} and \emph{just
make a move} mode. Essentially, depending on the time left on the board,
we would change the depth level. If there was sufficient time and it is
mid-game/late-game, the agent would choose a depth of 4, as it is in
\emph{Calm mode}. But as the time decreases, the agent too would have
less time to make a move and would search till lower depths. This idea
was inspired by real strategy game players, who often take more time in
the crucial part of the game when there is time remaining, but towards
the end have to speed up and often make mistakes due to the lack of
time. This brings the idea of dynamic time allocation which I
implemented.
\end{quote}

\hypertarget{dynamic-time}{%
\subsection{DYNAMIC TIME}\label{dynamic-time}}

\begin{quote}
Similar to dynamic deepening, dynamic time allocation per move is
essentially giving my agent an upper bound on how much time they can
spend making a certain move. The different modes explained above are
functions of how much time is left on the board. Table 1 shows the
different time allocations for different points in the game. During
\emph{just make a move} mode, the amount of time is a fraction of the
time remaining. When the minmax function reaches it's time allocation
for making a move, it is forced to return the best move it can find thus
far without searching any further. This way it ensures that the player
does not lose on time and makes a move even if it is a terrible one with
the hopes that they may beat the opponent if the other agent runs out of
time.
\end{quote}

\begin{longtable}[]{@{}
  >{\raggedright\arraybackslash}p{(\columnwidth - 2\tabcolsep) * \real{0.8749}}
  >{\raggedright\arraybackslash}p{(\columnwidth - 2\tabcolsep) * \real{0.1251}}@{}}
\toprule
\begin{minipage}[b]{\linewidth}\raggedright
\textbf{Conditions}
\end{minipage} & \begin{minipage}[b]{\linewidth}\raggedright
\textbf{Depth}
\end{minipage} \\
\midrule
\endhead
\textbf{turn \textless{} 4} & 3 \\
\textbf{turn \textgreater= 20 and num\_actions \textless= 90 and
time\_remaining\textgreater40} & 4 \\
\textbf{num\_actions \textless{} 10 and turn\_count\textgreater4} & 5 \\
\textbf{num\_actions \textless{} 50 and time\_remaining\textgreater10} &
4 \\
\textbf{num\_actions \textless{} 140 and time\_remaining\textgreater2} &
3 \\
\textbf{otherwise} & 2 \\
\bottomrule
\end{longtable}

\hypertarget{iterative-deepening-search}{%
\subsection{ITERATIVE DEEPENING
SEARCH}\label{iterative-deepening-search}}

\begin{quote}
Another idea explored was Iterative deepening, where it performs a BFS
for every level that is explored and keeps going a level deeper till it
runs out of allocated time for the move. The allocated time per move was
determined as a function of the time remaining and the stage (early,
middle, late) of the game the agent was at. The idea stemmed from
dynamic time allocation such that it can return the best move of depth 3
before going into depth 4 just in case we run out of time halfway
through depth 4.

The problem with this however was that the amount of time it took to
accomplish depth 3 was already large, and then rerunning it for depth 4
was a waste of time just to look one move ahead. Treating this as
another agent against dynamic depth and time would consistently make it
lose on time. To make it's time allocation better, I chose to reduce the
amount of time it gets per move, but this often cut short depth 4
search, and the extra time it took to do the depth 4 search was just a
waste as that search did not contribute and the best move of depth 3 was
returned. As each agent only gets 180 seconds in total, wasting time
like this was not viable and thus iterative deepening was dropped.
\end{quote}

\hypertarget{optimisations}{%
\section{OPTIMISATIONS}\label{optimisations}}

\hypertarget{alpha-beta-pruning}{%
\subsection{ALPHA-BETA PRUNING}\label{alpha-beta-pruning}}

\begin{quote}
To make the agent faster, alpha-beta pruning was used, which we will not
explain as it has already been covered in the course content. However,
alpha-beta pruning is dependent on how well the moves are sorted, which
can change the complexity to up to \(O(b^{\frac{d}{2}})\text{.\ }\)As b
is on average a large number (\textgreater50), half-ing \emph{d} could
be very beneficial and hence moves were pre-sorted.
\end{quote}

\hypertarget{pre-sorting-moves}{%
\subsection{PRE-SORTING MOVES}\label{pre-sorting-moves}}

\begin{quote}
Moves were sorted when the set of possible actions are being returned.
The two algorithms that were considered was Python's \(list.sort()\) and
\(sorted().\) Generally, \(list.sort()\ \)is 13\% faster and uses less
memory than \(sorted()\) (Dahlitz, 2019), with the trade-off that it is
not a stable algorithm. Luckily, the stability does not matter for the
moves as the initial order did not mean anything. The key used to
compare to moves was the greedy algorithm, where we apply the action and
see the evaluation of the board from that action only. While this is not
a perfect sort, it provides a good general idea of what a better move
might be and helps the pruning as such. The complexity of pre-sorting
moves is \(O(b\ log\ b)\), where \(b\) is the branching factor/ number
of actions.
\end{quote}

\hypertarget{move-pruning}{%
\subsection{MOVE PRUNING}\label{move-pruning}}

\begin{quote}
Another way to optimize is to reduce the total number of moves available
for the agent. While this does not change much, it will reduce the
number of moves by a bit which can cascade to be significant as the
depth increases. For example, we would never want to SPAWN at a cell
which neighbours the opponent, and hence those can be excluded.
Similarly, we would prefer SPAWN actions next to our cells, and
therefore those ones will be prioritized and returned.
\end{quote}

\hypertarget{bitboards-not-implemented}{%
\subsection{BITBOARDS (NOT
IMPLEMENTED)}\label{bitboards-not-implemented}}

\begin{quote}
BitBoards is another concept that was considered for our project. A
large portion of time consumed is taken by the \emph{deepcopy} of the
Board, which is stored as a dictionary. The average case copy complexity
of a dictionary us \(O(n)\), which is large when we have \(n = b^{d}\)
copies of the board. A solution to this is a hashed board or storing it
as a list, but due to the scope of the project this was not done.
Moreover, using a different data structure would require restructuring
of how possible moves are found and other functions, and thus not used.
This could have been a very powerful extension, however, as it enables
the use of memory tables as well to lookup predetermined board costs.
\end{quote}

\hypertarget{performance-evaluation}{%
\section{PERFORMANCE EVALUATION}\label{performance-evaluation}}

\begin{quote}
In order to test how well our agent was faring, we creating multiple
agents with slight variations and tested them locally to see which one
was the best. Initially we were testing with a minmax agent without any
optimisations, then created an alpha-beta pruned one and so on. After
multiple rounds, it was found that RED AGENT played best with depth 3,
while BLUE AGENT played best with a dynamic depth. In order to ensure it
never ran overtime with blue agent, the MAX\_TIME was updated, which
describes the runtime per move.

After the depths were configured, the next changes done were to the
evaluation function. We tried out multiple evaluation functions such as
number of tiles left, weighted power, total power, etc. Trying out
distance measures was also considered, but from Part A, it is known
finding distance is a fairly costly calculation and give the time and
space constraints, unless thoroughly examined and optimised, it would
not be feasible.
\end{quote}

\hypertarget{references}{%
\section{References}\label{references}}

Dahlitz, F. (2019). \emph{List sort vs sorted list}. Retrieved from
medium.:
ttps://medium.com/@DahlitzF/list-sort-vs-sorted-list-aab92c00e17\#:\textasciitilde:text=The\%20previous\%20investigations\%20showed\%20us\%2C\%20that\%20list.sort\%20is,use\%20list.sort\%2C\%20you\%20will\%20lose\%20your\%20original\%20list.

Foundation, P. S. (n.d.). \emph{Time Complexity}. Retrieved from Python:
https://wiki.python.org/moin/TimeComplexity

Wächter, L. (2021). \emph{Improving Minimax performance}. Retrieved from
gitconnected:
https://levelup.gitconnected.com/improving-minimax-performance-fc82bc337dfd

\end{document}
